\documentclass{beamer}
\usepackage[spanish]{babel}
\usepackage[utf8]{inputenc}
\usepackage{graphicx}

\newtheorem{introduccion}{Introducción}
\newtheorem{objetivos}{Objetivos}
\newtheorem{definicion}{Definición}
\newtheorem{ejemplo}{Ejemplo}

%%%%%%%%%%%%%%%%%%%%%%%%%%%%%%%%%%%%%%%%%%%%%%%%%%%%%%%%%%%%%%%%%%%%%%%%%%%%%%%
\title[Presentación con Beamer]{Trabajo final: Interpolación de Taylor.}
\author[Técnicas Experimentales]{Bolaños Florido,Cynthia. \\
Crus Guerra, Adrián.\\
Díaz Rodríguez, Diego}
\date[15-05-2013]{Miércoles 15 de marzo de 2013}
%%%%%%%%%%%%%%%%%%%%%%%%%%%%%%%%%%%%%%%%%%%%%%%%%%%%%%%%%%%%%%%%%%%%%%%%%%%%%%%

%\usetheme{Madrid}
%\usetheme{Antibes}
%\usetheme{boxes}
\usetheme{tree}
%\usetheme{classic}

\begin{document}

\begin{frame}

\includegraphics[width=0.15\textwidth]{img/ullesc.eps}
\hspace*{7.5cm}
\includegraphics[width=0.16\textwidth]{img/fmatesc.eps}
\titlepage

  \begin{scriptsize}
    \begin{center}
     Facultad de Matemáticas \\
     Universidad de La Laguna \\
     Grupo I.
    \end{center}
  \end{scriptsize}
\end{frame}


\begin{frame}
  \frametitle{Índice}  
  \tableofcontents[pausesections]
\end{frame}


\section{Motivación y objetivos.}
\begin{frame}
\frametitle{Motivación y objetivos.}
\begin{introduccion}
\LaTeX{} es un sistema de composición de textos, orientado especialmente a la creación de libros, 
documentos científicos y técnicos que contengan fórmulas matemáticas. 
\end{introduccion}
\begin{objetivos}
\LaTeX{} es un sistema de composición de textos, orientado especialmente a la creación de libros, 
documentos científicos y técnicos que contengan fórmulas matemáticas. 
\end{objetivos}
\end{frame}


\section{Fundamentos teóricos.}
\begin{frame}
\frametitle{Fundamentos teórcios.}
\begin{introduccion}
\LaTeX{} es un sistema de composición de textos, orientado especialmente a la creación de libros, 
documentos científicos y técnicos que contengan fórmulas matemáticas. 
\end{introduccion}
\begin{objetivos}
\LaTeX{} es un sistema de composición de textos, orientado especialmente a la creación de libros, 
documentos científicos y técnicos que contengan fórmulas matemáticas. 
\end{objetivos}
\end{frame}


\section{Procedimiento experimental.}
\begin{frame}
\frametitle{Procedimiento experimental.}
\begin{introduccion}
\LaTeX{} es un sistema de composición de textos, orientado especialmente a la creación de libros, 
documentos científicos y técnicos que contengan fórmulas matemáticas. 
\end{introduccion}
\begin{objetivos}
\LaTeX{} es un sistema de composición de textos, orientado especialmente a la creación de libros, 
documentos científicos y técnicos que contengan fórmulas matemáticas. 
\end{objetivos}
\end{frame}


\section{Conclusiones.}
\begin{frame}
\frametitle{Conclusiones.}
\begin{introduccion}
\LaTeX{} es un sistema de composición de textos, orientado especialmente a la creación de libros, 
documentos científicos y técnicos que contengan fórmulas matemáticas. 
\end{introduccion}
\begin{objetivos}
\LaTeX{} es un sistema de composición de textos, orientado especialmente a la creación de libros, 
documentos científicos y técnicos que contengan fórmulas matemáticas. 
\end{objetivos}
\end{frame}

\section{Bibliografia}
\begin{frame}
  \frametitle{Bibliografía}
  \begin{thebibliography}{10}
    \beamertemplatebookbibitems
    \bibitem[Fórmulas]{guia}  
   Fórmulas matemáticas
    {\small $http://eguia.ull.es/matematicas/query.php?codigo=299341201$}
    \beamertemplatebookbibitems
    \bibitem[URL: CTAN]{latex} 
    CTAN. {\small $http://www.ctan.org/$}
  \end{thebibliography}
\end{frame}


\end{document}









